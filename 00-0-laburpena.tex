\section*{Laburpena}
\setcounter{page}{1}
\pagenumbering{roman}
\fancyfoot[LE,RO]{\thepage}

\lettrine[lines=2, findent=3pt,nindent=0pt]{L}{an} hau metrologia koantikoaren baitan autoreak egindako hainbat ikerketa biltzen ditu.

Lehenengo kapituluan, Chapter~\ref{sec:in}, teknologia koantikoak duen garrantzia azpimarratzen da.

Bigarren kapituluan,

Hurrengo kapituloetan, autoreak garatu dituen ikerketa lanak azaltzen dira.
Lehenego eta behin, hirugarren kapituloan, Chapter~\ref{sec:vd}, polarizatu gabeko Dicke egoeren metrologia erabilgarritasuna aztertzen da.
Egoera koantiko puruak garatzea oso zaila da praktikoki, eta egoera nahasiak lortzen dira gehienbat laborategietan.
Arrazoi honegatik, autoreak egoera nahasi hauek identifikatzeko balio duen teknika bat garatu du.

Normalean, egoerak duen Fisher informazio koantikoa kalkulatzeko egoera osoaren deskripzio zehatza behar da.
