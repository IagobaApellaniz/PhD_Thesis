\section{Conclusions}
\thiswatermark{\put(1,-241){\color{l-grey}\rule{84pt}{48pt}}
\put(84,-241){\color{grey}\rule{410pt}{48pt}}}


\lettrine[lines=2, findent=3pt, nindent=0pt]{I}{n} this thesis we have presented some aspects of quantum metrology from three different perspectives.
Besides the introductory part, Chapters~\ref{sec:in} and \ref{sec:bg}, in Chapters~\ref{sec:vd}, \ref{sec:lt} and \ref{sec:gm}, our main results can be found.
In Chapter~\ref{sec:vd}, we have developed the theory of quantum metrology for metrology with noisy Dicke states.
In Chapter~\ref{sec:lt}, we have presented a method for witnessing the QFI with expectation values of some general observables.
Finally in Chapter~\ref{sec:gm}, we have computed precision bounds for gradient magnetometry.

The reader may notice that we have stressed the importance of bounding the quantum Fisher information based on some expectation values of some observables of the initial state.
As one can find in Chapters~\ref{sec:vd} and \ref{sec:lt}, we have pursued this idea.
It turns out that to compute the quantum Fisher information is not a trivial task and there is not measurement scheme to obtain it from the initial state apart from a complete tomography, which is a very demanding task for large system sizes.
Hence, some shortcuts to compute the bound of the QFI are necessary.

In Chapter~\ref{sec:vd}, we computed the precision bound for noisy unpolarized Dicke states based on some initial expectation values.
Moreover, we first reduced the number of expectation values needed to four.
More explicitly, we have to measure only the second and the fourth moments of the $y$-component and the $x$-component of the collective angular momentum in order to estimate the metrological usefulness of the system.
In practice, the fourth-order moments can also be well approximated by the second-order moments.
We note that after completing our calculations, we have recently became aware of a related work by Haine {\it et al} \cite{Haine2015}, which is based on the preliminary work in Ref.~\cite{Szigeti2014}.

In Chapter~\ref{sec:lt}, we developed a method based on the Legendre transform.
Based on this method, we are able to obtain a tight lower bound on the quantum Fisher information as a function of a set of expectation values of the initial state.
Furthermore, We tested our approach on extensive experimental data on photonic and cold gas experiments, and demonstrated that it works even for the case of thousands particles.
In the future, it would be interesting to use our method to test the optimality of various recent formulas giving a lower bound on the quantum Fisher information \cite{Zhang2014, Oudot2015}, as well as to improve the lower bounds for spin-squeezed states and Dicke states allowing for the measurement of more observables than the ones used in this publication.

On the other hand, in Chapter~\ref{sec:gm}, we have investigated the precision limits for measuring the gradient of a magnetic field with atomic ensembles arranged in different geometries and initialized in different states.
In particular, we studied spin-chain configurations as well as the case of two atomic ensembles localized at two different positions, and also the experimentally relevant set-up of a single atomic ensemble with an arbitrary density profile of the atoms was considered.
We discussed the usefulness of various quantum states for measuring the field strength and the gradient.
Some quantum states, such as singlet states, are insensitive to the homogeneous field.
Using these states, it is possible to estimate the gradient and saturate be Cramér-Rao bound, while for states that are sensitive to the homogeneous magnetic field, compatible measurements are needed for this task.
For spin chains and the two-ensemble case, the precision of the estimation of the gradient can reach the Heisenberg limit.
For the single ensemble case, only if strong correlation between the particles is allowed can the shot-noise limit be surpassed and even the Heisenberg limit be achieved.
However, even if the Heisenberg limit is not reached, single-ensemble methods can have a huge practical advantage compared to methods based on two or more atomic ensembles, since using a single ensemble makes the experiment simpler and can also result in a better spatial resolution.
