\lettrine[lines=2, findent=3pt,nindent=0pt]{I}{n} this section we will show some results regarding the metrological usefulness of some unpolarized states.
Recently it has been proven that among all states used for metrology the more polarized the state is the less margin to surpass the SQL the state has.
All together, one can also show that the unpolarized states are able to perform better than the polarized ones.
The more important figure of merit of such unpolarized but still useful states is the so-called unpolarized Dicke state, which consists of an equal number of particles pointing up and pointing down and it lives on the symmetric subspace.

One of the most particular features that this state has is that since it is a eigenstate of the collective operator $J_z$ with eigenvalue zero, it mus have a very large uncertainty on the operators $J_x$ and $J_y$. See Fig.

Since one cannot use the expectation value of the collective $J_l$ operators to see how the state evolve under the magnetic field, one have to go at least a new level and consider the evolution of the variances.

\subsection{Evolution of the expectation values}
Here we show how the expectation values of collective operators needed in this section evolve under the unitary dynamics because of the influence of the homogeneous magnetic field.

\be
  J_x(\Theta) = J_x \text{c}_\Theta - J_y \text{s}_{\Theta}
\ee
