\section*{Laburpena}
\label{sec:la}

\lettrine[lines=2, findent=3pt,nindent=0pt]{L}{an} honek metrologia kuantikoaren baitan egindako hainbat ikerketa biltzen ditu tesi moduan.
Aurkezten ditudan ikerketa hauek, Zientzia eta Teknologia Kuantikoko masterra bukatu ondoren, azkeneko lau urte hauetan Prof. Géza Tóth irakaslearen lan-taldean egindako lanen artean daude.
Tesi honetan agertzen ez diren beste hainbat lan plazaratu ditugu nik eta elkarrekin lan egin dugun hainbat ikertzaileek.
Publikatutako artikuluen lista tesi honen \pageref{sec:pu}. orrialdean topa daiteke.

Tesi honetan agertzen diren ikerketa lanak gauzatzeko ezin bestekoa izan da nazioarteko elkarlana.
Izandako elkarlanen artean Alemaniako Siegen hiriko unibertsitatean dagoen Otfried Gühnek zuzentzen duen TQO taldea dago.
Beste kolaborazio garrantzitsu bat Italiako Florentzian dagoen unibertsitateko Augusto Smerzik zuzentzen duen QSTAR taldea izan da.
Azkenik, Alemaniako Hannover hiriko unibertsitateko Carsten Klemptek zuzentzen duen ikerkuntza talde esperimental batekin izandako elkarlan emankorra aipagarria da baita ere.

\ref{sec:in}.~kapituluan teknologia kuantikoak eta metrologia kuantikoak duten garrantzia azpimarratzen dut.
Teknologia kuantikoa prozesu kuantikoez baliatzen baita klasikoki lortu ezin diren hainbat helburu lortzeko.
Adibide giza, ordenagailu kuantikoek hainbat posibilitate aldi berean aztertzeko izango luketen gaitasuna, edota simulazio kuantikoek modelo konplexu ezberdinak simulatzeko duten gaitasuna azpimarragarri dira nire ustez.

Beste motatako teknologi kuantikoak alde batera utzita, metrologia kuantikoan oinarritzen gara lan honetan.
Metrologia kuantikoak zenbatetsi nahi diren parametroen errorea txikiagotzeko aukera ematen du metrologia klasikoarekin alderatuz gero.
Klasikoki diseinatutako aparailu batek $N$ proba egin ondoren errorea $\sqrt{N}$ aldiz txikitzea lortzen duen bitartean, metrologia kuantikoa erabiliz errorea $N$ aldiz txikitu daitekeela aski ezaguna da \cite{Giovannetti2004, Paris2009}.
Muga gaindiezin hauei deritze "shot-noise scaling" aparatu klasikoen kasurako eta "Heisenber scaling" egoera kuantiko orokor batek gainditzerik ez duenarentzako.

Ezaguna da baita ere elkarlotura kuantikoak, mekanika kuantikoaren propietatea eta klasikoki azalpenik ez duenak, zenbatespenean duen garrantzia.
Elkarlotura kuantikoa zenbatespena hobetzeko ezinbestekoa da, aldiz, elkarlotura kuantiko mota guztiak ez dute balio errorea txikitzeko.
Elkarlotura kuantikoak eta metrologia kuantikoak duten erlazioaren hainbat azterketa hurrengo erreferentzietan topa daitezke \cite{Pezze2009, Louchet-Chauvet2010, Appel2009, Riedel2010, Gross2010, Luecke2011, Strobel2014, Hyllus2010}.

Honi guztiari gehitu behar zaio azken urteotan metrologia kuantikoak piztu duen arreta.
Kuantikak hobetutako metrologia erabiltzen da, adibidez, erloju atomikoetan \cite{Louchet-Chauvet2010, Borregaard2013, Kessler2014a}, zehaztasun handiko magnetometrian \cite{Wasilewski2010, Eckert2006, Wildermuth2006, Wolfgramm2010, Koschorreck2011, Vengalattore2007, Zhou2010}, edota uhin grabitazionalen detektagailuetan \cite{Schnabel2010, TheLIGOScientificCollaboration2011, Demkowicz-Dobrzanski2013}.

\ref{sec:bg}.~kapitulua metrologia kuantikoaren sarrera giza uler daiteke.
Bertan estatistikan oinarritutako hainbat kontzeptu azaltzen dira.
Estatistika datuetatik ondorio ulerkorrak ateratzeko erabiltzen den zientzia matematikoa da.
Adibidez, datu lagin baten batezbestekoa kalkulatzeko erabiltzen den prozedura,
edota datu lagin baten bariantza kalkulatzeko erabilitako formulak, azaltzen dira kapitulu honetan.
Datu lagin batek zenbatetsi nahi den parametroari buruzko informazioa izan ahal du bere baitan.
Adibide bezala, pilota bat bosgarren pisutik jaurtitzerakoan lurra jotzeraino igarotako denbora neurtu da hainbat aldiz.
Denbora ezberdin guzti hauek erabiliz, bosgarren pisuraino dagoen altuera kalkula daiteke grabitateak pilotarengan duen eragina aldez aurretik ezaguna denean.

Estatistikaren baitan kokatzen da datu lagin batetik zenbatespena egiterakoan saihestezina den errorearen muga klasikoaren kalkulua.
Muga hau Fisher informazioan, laginaren probabilitate distribuzio funtzioak eta zenbatetsi nahi den parametroaren aldaketaren arteko korrelazioa neurtzen duen kantitatean, oinarrituta dago.

Kontzeptu hauek aztertu ondoren, mekanika kuantikoko tesi honetan erabilitako hainbat tresna aurkezten dira.
Tresna eta definizio hauek tesi hau hobeto ulertzeko azaltzen dira tesiaren hasierako kapitulu honetan.
Egoera kuantikoaren definizio eta propietateak azaltzen dira, baita operadore kuantikoenak ere.
Egoera kuantikoa matrize baten bitartez irudikatu daiteke gehienetan.
Matrize honen karratuarekin egoera kuantiko bera lortzen bada, egoera purua dela esaten da.
Egoera kuantiko nahasiak aldiz, egoera puruen nahasketa baten bitartez adieraz daitezke.

Deskonposizio horien artean, egoera puruak beraien artean ortogonalak direnean deskonposizio propioa dela esaten da.
Egoera kuantiko bat beraz, deskonposizio propio baten bitartez adieraz daiteke $\rho\equiv\sum_\lambda p_\lambda \ketbra{\lambda}{\lambda}$, non $p_\lambda$ probabilitate bat eta $\ket{\lambda}$ egoera puru bat adierazten duten.
Partikula multzo baten aurrean gaudenean, aurreko propietateez gain, beste propietate interesante batzuk agertzen dira.
Elkarlotura kuantikoa, adibidez, partikula multzoetan definitzen da.
Multxo osoaren egoera kuantikoa banakorra ez denean elkarlotuta daudela esaten da.

Tesi honetako operadore erabilienak momentu angeluarraren osagaiak dira, bai partikula bakarraren momentu angeluarrarenak, baita partikula guztien momentu angeluar kolektiboarenak ere.
Momentu angeluarraren operadore hauek garrantzia handia daukate magnetometrian.
Partikula bakarraren spin operadoreak momentu angeluar operadoreak dira, eta spin operadoreen bitartez deskribatzen da partikulek eremu magnetikoekin daukaten interakzioa.
Oinarrizkotzat hartu daitekeen partikula bakarraren spin zenbakia gehienetan $\frac{1}{2}$ da.
Honi deitzen zaio \emph{qubit} kuantikoa.

Bestalde, momentu angeluarraren operadoreek, deskonposizio propioan, base berri bat sortzen dute.
Base honetan hainbat egoera kuantiko berezi topa daitezke, esaterako, Dicke egoera simetrikoak, edota singletea.
Egoera hauek aztertzerakoan ikusten da magnetometrian edota beste teknologia kuantikoetan duten erabilgarritasuna.

Estatistikaren zenbatespen metodologia eta mekanika kuantikoa batzean, metrologia kuantikoari buruz hitz egiten gaude.
Metrologia kuantikoan zenbatespen prozesuaren errorearen mugak aztertzen dira askotan.
Tesi honetan aurkeztutako ikerkuntzekin muga hauen bilaketan aurrera pausu garrantzitsuak ematen dira.
Fisher informazio kuantikoa da normalean erabiltzen den tresnarik esanguratsuena.
Zenbatespena egiteko erabiltzen den egoera kunatikoan eta interakzioak sortzen duen egoeraren eboluzioan oinarrituta dago Fisher informazio kuantikoa.
Beraz, ezinbestekoa da hasierako egoera kuantikoa ezagutzea Fisher informazio kuantikoa kalkulatzeko.
Hurrengo paragrafoetan azaltzen dira, aldez aurretik egoera kuantikoa zein den jakin gabe, muga hauek bilatzeko garatu ditugun tresnak.

\ref{sec:vd}.~kapituluan beraz, lehenbiziko ikerketa lana aurkezten dut: Polarizatu gabeko Dicke egoeratik hurbil dauden egoera kuantiko nahasiek metrologian duten erabilgarritasuna.
Egoera kuantiko puruak gauzatzea oso zaila da praktikoki, eta egoera nahasiak lortzen dira gehienbat laborategietan.
Arrazoi honegatik, kapitulu honetan egoera nahasi hauek metrologian duten erabilgarritasunaren arabera sailkatzeko balio duen teknika aurkezten da.

Egoera ez polarizatuak egoera polarizatuak baino erabilgarriagoak izan daitezke magnetometrian.
Egoera polarizatuak erabiltzerakoan aldiz, eremu magnetikoaren magnitudea zenbatestea nahiko zuzena da.
Egoerak denbora tarte batean eremu magnetikoaren pean polarizazioan jasandako  errotazioa neurtzen da eta aldaketa honetatik eremu magnetikoaren zenbatespena egiten da.
Bestalde, egoera ez polarizatuak ezin dute teknika hau erabili, nahiz eta Fisher informazio kuantikoa kalkulatzerakoan magnetometriarako erabilgarriagoak direla argi dagoela ikusi.

Hau dela eta, Dicke egoera ez polarizatuek duten beste propietate bat erabiltzen da, polarizazioaren sakabanaketa.
Propietate hau polarizazioaren neurketetan lortzen den datuen sakabanaketa da.
Datuen sakabanaketa hau Heisenbergen ziurgabetasun printzipioarekin lotuta dago.
Dicke egoera ez polarizatuetan sakabanaketa hau txiki izatetik $N^2$-ko proportzioetara heltzen den magnitude bat da, beraz, $N^2$-ko proportzioetako aldaketa neurtuko da.
Polarizazioan oinarritutako zenbatespenak $N$-ko proportzioetara heltzen diren bitartean, sakabanaketan oinarritutakoak zenbatespenak "Heisenberg scaling" muga fisikotik hurbilago daude, ikusi \pageref{fig:in-magnetometry-totally-polarized}. orrian dagoen \ref{fig:in-magnetometry-totally-polarized}.~irudia.
Polarizazioa gezi gorriak ematen duen bitartean, ziurgabetasun zirkulu urdinak ematen du.

Guzti honetan oinarrituta, polarizazioaren sakabanaketaren aldaketa neurtzerakoan eremu magnetikoaren zenbatespena egin ahal da.
\ref{sec:vd}.~kapituluan erroreen hedapenaren formula aplikatuz, zenbatespenean gertatuko den errorea kalkulatzen dugu.
Errore hau hasierako egoeraren itxarotako balioen funtzio bezala idatzi ondoren, aski da lau behagarriren itxarotako balioak neurtzea.
Lau balore hauek neurtzearekin batera zenbatespenaren errorea lortuko dugu.
Errore hau Fisher informazio kuantikoaren gainetik egon arren, egoerak sailkatzen laguntzen du beraz.
Gainera, lau operadorereen neurketa partikula asko duten egoeren tomografia egitea baino askoz errazagoa da esperimentalki.
Kapitulua bukatutzat emateko, errorearen formulan oinarrituta, are gehiago sinplifikatzen dugu formula hau.
Oraingoan, operadore biren itxarotako balioan oinarritzen den beste ordezko ekuazio bat aurkezten dugu, honek dakarren abantaila esperimentala azpimarratuz.

\ref{sec:lt}.~kapituluan, Fisher informazio kuantikoaren mugak aztertzen ditugu egoera kuantiko baten operadore ezberdinek daukaten itxarotako balioen funtzio bezala.
Beraz, arazo berdinari egiten zaio aurre kapitulu honetan.
Praktikoki egoera kuantikoa zehatz-mehatz jakitea ezinezkoa denez, eta are gutxiago partikula asko duten sistemetan, Fisher informazio kuantikoa mugatzerakoan behagarriek egoera kuantikoan duten itxarotako balioetan oinarritzen gara berriz ere.

Oraingoan aldiz, problema honi beste ikuspuntu batekin aztertzeari ekiten diogu.
Legendreren transformazioan oinarritutako elkarloturaren neurketak egiteko metodo baten oinarrituta \cite{Guehne2007}, Fisher informazio kuantikoaren doitutako mugak topatzera heltzen gara.

Kapituluan zehar hainbat adibide garatzen dira.
Metodo honek edozein behagarri hartu eta beraren itxarotako balio bera duten egoera kuantiko guztien artean Fisher informazio kuantiko baxuenekoa aukeratzea ahalbidetzen du.
Metodo honek, bat bakarra beharrean, hainbat behagarri hartu ditzake.
Beraz, hainbat behagarri sorta hartu eta beraien itxarotako balioak aldez aurretik dakizkigula, metodoak emandako Fisher informazio kuantikoaren muga ezberdinak aztertzen ditugu.

Azkenik, adibide konkretu batzuei jarraituz, gure metodoa partikula askotako egoeretara nola luzatu daitekeen aztertzen dugu.
Datu esperimentalak erabiliz, aldez aurretik egindako esperimentuetarako zenbatespenaren mugak kalkulatzen ditugu.
Datu hauek \cite{Luecke2014} eta \cite{Gross2010}~erreferentziei jarraituz lortu ditugu.

Tesi honetan aurkezten dudan azkenengo ikerketa lana \ref{sec:gm}~kapituluan topa daiteke: Eremu magnetikoaren gradientearen zenbatespenaren mugak atomo multzoak erabiltzerakoan.
Eremu magnetikoaren gradientea eremu magnetikoak espazioan daukan aldaketa adierazten du.
Aurreko kapituluetan ez bezala, honetan, Fisher informazio kuantikoa kalkulatzen da.

Eremu magnetikoa, beraz, parametro birekin zehaztuta dago, eremu magnetikoaren parte homogeneoa eta gradientea.
Ondorioz, parametro bat baino gehiago zenbatetsi behar ditugu, nahiz eta gradiente parametroan soilik interesatuta egon.
Parametro bat baino gehiagoko metrologia kuantikoko oinarrizko problematzat hartu daiteke eremu magnetikoaren gradientearen zenbatespena.

Eremua magnetikoaren gradientea kalkulatzeko ezinbestekoa da egoera kuantikoak espazioan daukan izaera aztertzea, hau da, egoera kuantikoak espazioa nola betetzen duen jakitea.
Egoera kuantikoaren espazioaren partea, partikula puntualez osatutako egoera batera sinplifikatzen dugu, nahiz eta gure emaitzak bestelako kasuetara ere egokitzen den.

Adibidez, lehenengo kasuan atomoak espazioko puntu ezberdinetan jartzen dira ilara zuzen bat sortuz.
Atomo ezberdinek eremu magnetikoaren intentsitate ezberdinak sumatuko dituzte.
Spin egoeraren arabera beraz, Fisher informazio ezberdinak kalkulatzen ditugu.
Bigarren kasuan atomo guztiak espazioko bi puntu ezberdinetan kokatuta daude, atomoen erdia puntu batean eta beste erdia bestean.
Kasu honetan topa daiteke eremu magnetikoaren zenbatespenerako spin egoerarik onena, Heisenbergen printzipioez mugatutako zenbatespena ematen duena.

Azkenengo kasuan, atomoak espazioan zehar era desordenatu baten sakabanatuta daude.
Esperimentu askotan topa daitekeen egoera da hau.
Adibidez, atomoak barrunbe batean daudenean.
Spin egoera ezberdinak aztertzen ditugu eta kasu bakoitzean beraien Fisher informazioa, zenbatespenean duten muga teorikoa, kalkulatzen dugu.

Ondorio gisa, lan honetan aurkeztutako azterketek zenbatespen kuantikoa dute jomuga.
Lehenengo ikerkuntza bietan, esperimentuen konplexutasuna sinplifikatzen dute. Egoera kuantikoan oinarritu beharrean, behagarri batzuen itxarotako balioetan oinarritzen baita zenbatespenaren errorearen muga.
Metodo hauen inplementazio praktikoak aztertu ditugu aldez aurretik egindako esperimentuen datuak erabiliz.
Honek guztiak, etorkizunean egingo diren metrologia kuantikoko esperimentuetan, gure metodoak erabiltzea errazten du.
Bestalde, eremu magnetikoaren gradientearen azterketan topatu ditugun muga teorikoak Heisenbergen proportzionaltasuna ahalbidetzen dute.
Proportzionaltasun hau bi partikula multzo erabiltzen direnean eta baita multzo bakarra erabiltzen denean ere agertu daitekeela erakutsi dugu.
Partikula multzo bakarra erabiltzerakoan beraz, partikula zenbakiarekin batera txikitzen da errorea, esperimentua eta ondoren etor daitekeen inplementazio praktikoa asko sinplifikatuz, eta Heisenbergen proportzionaltasuna oraindik ere mantenduz.
