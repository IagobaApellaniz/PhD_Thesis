\section*{Laburpena}
\label{sec:la}

\lettrine[lines=2, findent=3pt,nindent=0pt]{L}{an} honek metrologia kuantikoaren baitan egindako hainbat ikerketa biltzen ditu.
Zientzia eta Teknologia Kuantikoko masterra bukatu ondoren eta azkeneko lau urte hauetan Prof. Géza Tóth irakaslearen lan-taldean egindako ikerketa lanak batzen ditut ain zuzen.
Tesi honetan agertzen ez diren beste hainbat lan plazaratu ditugu baita ere.
Orainarte ni autore bezala agertzen naizen artikuluen lista tesi honen \pageref{sec:pu}. orrialdean topa daiteke.

Tesi honetan agertzen diren ikerketa lanak gauzatzeko ezin bestekoak izan dira kolaborazio internazionalak.
Izandako elkarlanen artean Alemaniako Siegen hiriko unibertsitatean dagoen Otfried Gühnek zuzentzen duen TQO taldea da bat.
Beste kolaborazio garrantsitsu bat Italiako Florentzian dagoen unibertsitateko Augusto Smerzik zuzentzen duen QSTAR taldea izan da.
Azkenik, Alemaniako Hannover hiriko unibertsitateko Carsten Klemptek zuzentzen duen ikerkutza talde experimental batekin izandako elkarlan emankorra aipagarria da baita ere.

\ref{sec:in}.~kapituluan teknologia kuantikoak eta metrologia kuantikoak duten garrantzia azpimarratzen dut.
Teknologia kuantikoa prozesu kuantikoez baliatzen bait da klasikoki egin ezin diren hainbat helburu lortzeko.
Adibide giza, ordenagailu kuantikoek hainbat posibilidade aldi berean aztertzeko izango luketen gaitasuna, edota simulazio kuantikoek modelu konplexu ezberdinak simulatzeko duten gaitasuna azpimagarri dira nire ustez.

Oraingoan, beste motatako teknologi kuantikoak alde batera utzita, metrologia kuantikoak zenbatetsi nahi diren parametroen errorea txikiagotzeko aukera ematen du metrologia klasikoarekin alderatuz gero.
Klasikoki diseinatutako aparailu batek $N$ proba egin ondoren errorea $\sqrt{N}$ aldiz txikitzea lortzen duen bitartean, metrologia kuantikoa erabiliz errorea $N$ aldiz txikitu daitekeela aski ezaguna da \cite{Giovannetti2004, Paris2009}.
Muga gaindiezin hauei deritze "shot-noise scaling" aparatu klasikoen kasurako eta "Heisenber scaling" egoera kuantiko orokor batek gainditzerik ez duenarentzako.

Esaguna da baita ere elkarlotura kuantikoak, mekanika kuantikoaren propietatea eta klasikoki azalpenik ez duenak, zenbatespenean duen garrantsia.
Elkarlotura kuantikoa zenbatespena hobetzeko ezinbestekoa da, aldiz, elkarlotura kuantiko mota guztiak ez dute balio errorea txikitzeko.
Elkarlotura kuantikoak eta metrologia kuantikoak duten erlazioaren azterketa hurrengo erreferentzietan topa daiteke \cite{Pezze2009, Louchet-Chauvet2010, Appel2009, Riedel2010, Gross2010, Luecke2011, Strobel2014, Hyllus2010}.

Honi guztiari gehitu behar zaio azken urteotan metrologia kuantikoak piztu duen arreta.
Kuantikak hobetutako metrologia erabiltzen da, adibidez, erloju atomikoetan \cite{Louchet-Chauvet2010, Borregaard2013, Kessler2014a}, zehaztasuneko haundiko magnetometrian \cite{Wasilewski2010, Eckert2006, Wildermuth2006, Wolfgramm2010, Koschorreck2011, Vengalattore2007, Zhou2010} edota uhin grabitazionalen detektagailuetan \cite{Schnabel2010, TheLIGOScientificCollaboration2011, Demkowicz-Dobrzanski2013}.

\ref{sec:bg}.~kapituluan metrologia kuantikoaren sarrera giza uler daiteke.
Bertan estatistikan oinarritutako hainbat kontzeptu azaltzen dira.
Estatistika, datuetatik ondorio ulerkorrak ateratzeko erabiltzen den zientzia matematikoa da.
Adibidez, datu lagin baten batabestekoa kalkulatzeko erabiltzen den prozedura,
edota datu lagin baten bariantza kalkulatzeko erabilitako formulak, azaltzen dira kapitulu honetan.
Datu lagin batek zenbatetsi nahi den parametroari buruzko informazioa izan ahal du bere baitan.
Adibidez bezala, pilota bat bostgarren pisutik jaurtitzerakoan lurra jotzeraino igarotako denbora neurtu da hainbat aldiz.
Denbora ezberdin guzti hauek erabiliz, bostgarren pisuraino dagoen altuera kalkula daiteke grabitateak pilotarengan duen eragina aldez aurretik esaguna denean.

Estatistikaren baitan kokatzen da datu lagin batetik zenbatespena egiterakoan sahietsezina den errorearen muga klasikoaren kalkulua.
Muga hau Fisher informazioan, laginaren probabilitate distribuzio funtzioak eta zenbatetsi nahi den parametroaren aldaketaren arteko korrelazioa neurtzen duen kantitatean, oinarrituta dago.

Kontzeptu hauek aurkeztu ondoren, mekanika kuantikoko hainbat tresnen sarrera bat dator.
Tresna eta definizio hauek tesi hau hobeto ulertzeko eta aurrerago erabiltzen direlako azaltzen dira tesiaren hasierako kapitulu honetan.
Egoera kuantikoaren definizio eta propietateak azaltzen dira, baita operadore kuantikoenak ere.
Egoera koantikoa matrize baten bitartez irudika daiteke gehienetan.
Matrize honen karratuarekin egoera kuantiko bera lortzen bada, egoera purua dela esaten da.
Egoera kuantiko nahasiak aldiz, egoera puruen nahasketa baten bitartez adieraz daitezke.
Deskonposizio horien artean, egoera puruak beraien artean ortogonalak izatearena aurkitzen da.
Egoera kuantiko bat beraz, deskonposizio ortogonal baten bitartez adieraz daiteke baita ere $\rho\equiv\sum_\lambda p_\lambda \ketbra{\lambda}{\lambda}$, non $p_\lambda$ probabilitate bat eta $\ket{\lambda}$ egoera puru bat adierazten duten.
Partikula multxo baten egoera koantikoan, aurreko propietateez gain, beste propietate interesante batzuk agertzen dira.
Elkarlotura kuantikoa, adibidez, partikula multxoetan definitzen da.
Multxo osoaren egoera kuantikoa banakorra ez denean elkarlotuta daudela esaten da.

Tesi honetako operadore erabilienak momentu angeluarraren osagaiak dira, bai partikula bakarraren momentu angeluarrarenak baita partikula guztien momentu angeluar kolektiboarenak ere.
Momentu angeluarraren operadore hauek garrantzia haundia daukate magnetometrian.
Partikula bakarraren spin operadoreak ere momentu angeluar operadoreak dira, eta spin operadoreen bitartez deskribatzen da partikulek eremu magnetikoekin daukaten interakzioa.
Onarrizkotzat hartu daitekeen partikula bakoitzaren spin zenbakia gehienetan $\frac{1}{2}$ da.
Honi deitzen zaio \emph{qubit} kuantikoa.

Bestalde, momentu angeluarraren operadoreak osagai ortogonalez osatuta daudenez, base berri bat sortzen dute.
Base honetan hainbat egoera kuantiko berezi topa daitezke, esaterako, Dicke egoera simetrikoak, edota singletea.
Egoera hauek aztertzerakoan ikusten da magnetometrian edota beste teknologia kuantikoetan duten erabilgarritasuna.

Estatistikaren zenbatespen metodologia eta mekanika kuantikoa batzean, metrologia kuantikoaz hitzegiten gaude.
Metrologia kuantikoan zenbatespen prozesuaren errorearen mugak aztertzen dira askotan, eta tesi honetan ikertutakoarekin muga hauen bilaketan aurrera pausu garrantsitsuak ematen dira.
Muga hauen bilaketan Fisher informazio kuantikoa da erabiltzen den tresnarik esanguratsuena.
Zenbatespena egiteko erabiltzen den egoera kunatikoan eta interakzioak sortzen duen egoeraren eboluzioan oinarrituta dago Fisher informazio kuantikoa.
Beraz, ezinbestekoa da hasierako egoera kuantikoa ezagutzea Fisher informazio kuantikoa kalkulatzeko.
Hurrengo paragrafoetan azaltzen dira, aldez aurretik egoera kuantikoa zein den jakin gabe, muga hauek bilatzeko garatu ditugun tresnak.

\ref{sec:vd}.~kapituloan beraz, lehenbiziko ikerketa lana aurkezten dut: Polarizatu gabeko Dicke egoeratik hurbil dauden egoera kuantiko nahasiek metrologian duten erabilgarritasuna.
Egoera kuantiko puruak gauzatzea oso zaila da praktikoki, eta egoera nahasiak lortzen dira gehienbat laborategietan.
Arrazoi honegatik, kapitulo honetan egoera nahasi hauek metrologian duten erabilgarritasunaren arabera sailkatzeko balio duen teknika bat aztertzen da.

Egoera ez polarizatuak egoera polarizatuak baino erabilgarriagoak izan daitezke magnetometrian.
Egoera polarizatuak erabiltzerakoan aldiz, eremu magnetikoaren magnitudea neurtzea nahiko zuzena da.
Egoerak denbora tarte batean eremu magnetikoarenpean polarizazioan jasandako  rotazioa neurtzen da eta aldaketa honetatik eremu magnetikoaren zenbatespena egiten da.
Bestalde, egoera ez polarizatuak ezin dute teknika hau erabilitzen, nahiz eta Fisher informazio koantikoa kalkuratzerakoan magnetometriarako erabilgarriagoak direla argi dagoela ikusi.

Hau dela eta, Dicke egoera ez polarizatuek duten beste propietate bat erabiltzen da, polarizazioaren zakabanaketa.
Propietate hau polarizazioaren neurketetan lortzen den datuen zakabanaketa da.
Ain zuzen, datuen zakabanaketa hau Heisenbergen zihurgabetasun printzipioarekin lotuta dago.
Zakabanaketa hau txikia izatetik $N^2$-ko proportzioetara heltzen den magnitude bat da Dicke egoera ez polarizatuetan, beraz $N^2$-ko proportzioetako aldaketa neurtuko da.
Polarizazioan oinarritutako zenbatespenak $N$-ko proportzioetara heltzen diren bitartean, zakabanaketan oinarritutakoak zenbatespenak "Heisenberg scaling" muga fisikotik hurbilago daude, ikusi \pageref{fig:in-magnetometry-totally-polarized}. orrian dagoen \ref{fig:in-magnetometry-totally-polarized}.~irudia.
Polarizazioa gezi gorriak ematen duen bitartean, zihurgabetazun zirkulu urdinak ematen du.

Guzti honetan oinarruturik, polarizazioaren zakabanaketaren aldaketa neurtzerakoan eremu magnetikoaren zenbatespena egin ahal da.
Erroreen hedapenaren formula aplikatuz, zenbatespenean gertatuko den errorea kalkulatzen dugu \ref{sec:vd}.~kapituluan.
Errore hau hasierako egoeraren funtzio bezala idatzi ondoren, lau operadore hasierako egoeran duten itxarotako balioak neurtuz kalkula daiteke.
Lau balore hauek neurtzearekin batera zenbatespenaren errorea kalkulatuko dugu.
Errore hau Fisher informazio kuantikoaren gainetik egon arren, egoerak zailkatzen laguntzen du baita ere.
Gainera, lau operadoereen neurketa partikula askodun egoeraren tomografia egitea baino askoz errazagoa da experimentalki.
Kapituloa bukatutzat emateko, errorearen formulan oinarrituta, are gehiago sinplifikatzen dugu formula hau, oraingoan operadore biren itxarotako balioan oinarretzen den beste ekuazio alternatibo batera, honek dakarren abaintaila experimentala azpimarratuz.

\ref{sec:lt}.~kapituloan, Fisher informazio kuantikoaren mugak aztertzen ditugu egoera kuantiko baten operadore ezberdinek daukaten itxarotako balioen funtzio bezala.
Arazo berdinari egiten zaio aurre kapitulu honetan.
Praktikoki egoera kuantikoa zehatz-mehatz jakitea ezinezkoa denez, eta are gutxiago partikula askodun sistemetan, Fisher informazio kuantikoa mugatzerakoan behagarriek egoera kuantikoan duten itxarotako balioetan oinarritzen gara.

Oraingoan aldiz, problema honi beste enfoke batez aztertzeari ekiten diogu.
Legendreren transformazioan oinarritutako elkarloturaren neurketak egiteko metodo baten oinarrituta \cite{Guehne2007}, Fisher informazio kuantikoaren doitutako mugak topatzera heltzen gara.

Honen ondoren, kapituloan zehar, ainbat adibide garatzen dira.
Metodo honek edozein behagarri hartu eta beraren itxarotako balioa duten egoera kuantiko guztien artean Fisher informazio kuantiko baxuenekoa aukeratzea ahalbidetzen du.
Metodo honek era berean hartu ditzazke ainbat behagarri bat bakarra beharrean.
Beraz, ainbat behagarri sorta hartu eta beraien itxarotao balioak aldez aurretik dakizkigula, metodoak emandako Fisher informazio kuantikoaren muga ezberdinak aztertzen ditugu.

Azkenik, gure metodoa partikula askotako egoeretara nola luzatu aztertzen dugu adibide konkretu batzuei jarraituz.
Baita datu experimental errealak aplikatu ere.
Datu hauek \cite{Luecke2014} eta \cite{Gross2010}~erreferentziei jarraituz lortu ditugu.

Tesi honetan aurkezten dudan azkenengo ikerketa lana \ref{sec:gm}~kapituluan topa daiteke: Eremu magnetikoaren gradientearen zenbatespen mugak atomo multzoak erabiltzerakoan.
Eremu magnetikoaren gradientea eremu magnetikoaren espazioan duen aldaketa adierazten du.
Aurreko kapituloetan ez bezala, honetan Fisher informazio kuantikoa kalkulatzen da.

Eremu magnetikoa beraz parametro birekin zehaztuko litzateke, eremu magnetikoaren parte homogeneoa eta gradientea.
Beraz, parametro bat baino gehiago zenbatetsi behar ditugu.
Parametro bat baino gehigagoko metrologia kuantikoko onarrizko problematzat hartu daiteke eremu magnetikoaren gradientearen zenbatespena.

Eremua magnetikoaren gradientea kalkulatzeko ezinbestekoa da egoera kuantikoak espazioan daukan izaera aztertzea, hau da, egoera kuantikoak espazioa nola betetzen duen jakitea.
Egoera kuantikoaren espazioaren partea partikula puntualez osatuta egoera batera sinplikatzen dugu nahiz eta gure emaitzak bestelako kasuetara ere egokitzen den.

Adibidez, lehenengo kasuan atomo bana ilaran espazioko puntu ezberdinetan jartzen dira.
Atomo ezberdinek eremu magnetiko intentzitate ezberdinak zumatuko dute.
Spin egoeraren arabera beraz, Fisher informazio ezberdinak kalkulatuko ditugu.
Bigarren kasuan atomo guztiak espazioko bi puntu ezberdinetan kokatuta daude, atomoen erdia puntu batean eta beste erdia bestean.
Kasu honetan topa daiteke spin egoera bat eremu magnetikoaren zenbatespenerako onena dena, Heisenbergen printzipioez mugatutako zenbatespena lortzen duena.

Azkenengo kasuan, atomoak era desordenatu batean zakabanatuta daude.
Expermentu askotan topa daitekeen egoera da hau, atomoak barrunbe batean daudelarik.
Spin egoera ezberdinak aztertzen ditugu eta kasu bakoeitzean beraien Fisher informazioa, zenbatespenean duten muga teorikoa, kalkulatzen dugu.

Ondorio gisa, lan honetan aurkeztutako azterketek zenbatespen kuantikoa dute jomuga.
Lehenego ikerkuntza bietan, experimentuen konplexitatea sinplifikatzen dute. Egoera kuantikoan oinarritu beharrean, behagarri batzuen itxarotako balioetan oinarritzen bait da zenbatespenaren errorearen muga.
Metodo hauen inplementazio praktikoak aztertu ditugu baita ere, aldez aurretik egindako experimentuen datuak erabiliz.
Honek guztiak, etorkizunean egingo diren metrologia kuantikoko experimentuetan, gure metodoak erabiltzea errazten du.
Bestalde, eremu magnetikoaren gradientearen azterketan topatu ditugun muga teorikoak Heisenbergen proportzionaltasuna ahalbidetzen dute.
Proportzionaltasun hau bi partikula multxo erabiltzen direnean eta baita multxo bakarra erabiltzen denear ere ager daitekeela erakutzi dugu.
Partikula multxo bakarra erabiltzerakoan beraz, partikula zenbakiarekin bate txikitzan da errorea, experimentua eta ondoren etor litekeen inplementazio praktikoa asko sinplifikatuz eta Heisenbergen proportzionalitatea oraindik ere mantenduz.
